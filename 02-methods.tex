\section*{Methods}

M\'{y}vatn has a tundra-subarctic climate, and the surrounding landscape is
dominated by heathland and grassland vegetation \citep{Einarsson2004}.
Arthropod samples were collected approximately every 15d from five sites around the
shoreline of the lake from May--August in 2008--2019.
Each site included 3--4 plots at various distances
(5m, 50m, 150m, and 500m) along a transect perpendicular to the lakeshore
(some transects only extended to 150m).
Midge deposition was sampled at each site using aerial infall traps \citep{Dreyer2015},
and activity-density of ground-dwelling arthropods was sampled using pitfall traps
\citep{Southwood2009}.
Activity-density is a measurement of relative abundance, which represents a combination
of both population size and behavioral phenomena, both of which potentially respond
to the allochthonous inputs of resources \citep{Ostfeld2000}.
For this analysis, we used the cumulative May--August catch of predatory and
detritivorous arthropods (Figure \ref{fig:obs-data}),
including ground beetles (Carabidae), rove beetles (Staphylinidae),
harvestmen (Opiliones), ground spiders (Gnaphosidae),
sheet-weaving spiders (Linyphiidae), and wolf spiders (Lycosidae).



We analyzed taxon-level variation using an extension of
linear mixed models.
Our model included (1) fixed effects for continuous predictor variables 
(number of years since initial sampling year, distance from the lake, and midge deposition);
(2) random intercepts grouped by taxon, taxon $\times$ site, and taxon $\times$ plot; and
(3) random slopes for the three predictor variables grouped by taxon.
The fixed effects give the average response across taxa to the predictor variables,
while the corresponding random effects give the deviation for each taxon from this
average response \citep{Jackson2012}.
The overall response of a given taxon is the sum of the fixed and random components.
Quantifying responses for each taxon in a single model 
allows for "shrinkage" or “partial pooling” of the taxon-specific responses towards the mean response,
which reduces the noisiness of individual estimates and ameliorates concerns of multiple
comparisons \citep{Gelman2012} that have been raised for the examination of
responses for many taxa separately \citep{Mcgarigal2013}.
We log-transformed midge deposition and distance 
before z-scoring all of the predictors
by either subtracting the mean (midges, distance) or the minimum (time)
and then dividing by the standard deviation.
For time we subtracted the minimum so that the intercepts 
quantified the initial abundance of each taxon in each plot.



To formulate the autoregressive component of the model, we defined a vector $\mathbf{y}$
consisting of transformed counts for each taxon in each plot through time.
Because population processes are generally multiplicative, we log-transformed the counts,
adding one to all values to accommodate the modest number of zeros.
We then z-scored all values to improve interpretability of the model coefficients.
The rows of $\mathbf{y}$ were grouped as individual time series
(i.e., taxon--plot combinations), with observations within each group ordered by time.
For each time series, we then specified an autoregressive state-space model 
for the $i$th observation of $\mathbf{y}$ as
%
\begin{equation} \label{eq:y-i}
\begin{split}
    z_i &= \mathbf{x}_i^\text{T} {\boldsymbol\beta}_i +
        \left( \phi_{\text{taxon}[i]} \right)^{\text{time}_i - \text{time}\sp{\prime}_i}
        z\sp{\prime}_i + \varepsilon_i \\
    y_i &\sim \mathcal{N} \left(z_i, \; \sigma_{\text{obs}} \right)
    \text{,}
\end{split}
\end{equation}
%
\noindent where $z_i$ and $z\sp{\prime}_i$ are latent and lagged-latent abundances on the transformed scale,
$\mathbf{x}_i^\text{T}$ is the transposed vector
of predictor values for the $i$th observation,
${\boldsymbol\beta}_i$ is a vector of coefficients 
modeled hierarchically according to the fixed and random effects structure described above,
$\phi_{\text{taxon}[i]}$ is the taxon-specific autoregressive (AR) parameter,
$\text{time}_i$ is the current time (in years),
$\text{time}\sp{\prime}_i$ is the lagged time,
$\varepsilon_i$  is the Gaussian process error 
with standard deviation $\sigma_{\text{proc}}$,
and $\sigma_{\text{obs}}$ is the observation error standard deviation.
Because $y_i$ consists of log-transformed abundances, equation \ref{eq:y-i} is
essentially a log-linear model of population dynamics, with the population growth rate given by
$\mathbf{x}_i^\text{T} {\boldsymbol\beta}_i$.
When $\phi_{\text{taxon}[i]} = 0$,
the dynamics are determined primarily by $\mathbf{x}_i^\text{T} {\boldsymbol\beta}_i$,
indicating strong population regulation associated with contemporary environmental conditions \citep{Ives2010, Ziebarth2010}.
In contrast, $\phi_{\text{taxon}[i]}$ near 1 implies weak regulation as there is strong 
``memory'' of previous population states (i.e. high temporal autocorrelation).
Exponentiating the elapsed time with base $\phi_{\text{taxon}[i]}$ allows the model
to accommodate unequal time steps \citep{Zuur2009}.
The autoregressive term in equation (\ref{eq:y-i}) was dropped for the initial value of each time series,
such that the initial value was given entirely by predictor variables and associated coefficients.
The primary difference between the present approach and more traditional linear models
with AR structures \citep[e.g.,][]{Zuur2009} is the estimation of separate values of the AR
parameter for different groups (i.e., taxa).
This is an important extension, as it allows members of a community with
different levels of temporal autocorrelation to be included in a single model.



Community analyses often utilize dimension reduction to characterize 
variation in community composition \citep{Mcgarigal2013}.
However, with autoregressive mixed models it is possible to infer variation in community composition 
directly from the variation in taxon-specific responses.
We illustrated this in two ways.
First, we examined the posterior distributions of the fixed and random effects 
associated with the time, distance, and midge deposition. 
Fixed effects characterize the mean responses among taxa, 
and substantial overlap with zero would indicate the absence of a consistent community-wide 
response to a given predictor.
Random effects characterize variation in the responses among taxa,
which is the basis for changes in composition per se. 
While random effect standard deviations cannot overlap zero,
concentration of posterior density near the zero-boundary 
would imply limited variation in community composition in response to a given predictor.
Second, we used principal components analysis on fitted values from the model
to generate orthogonal axes
characterizing the expected variation in the community due to the predictors
\citep[similar to][]{Jackson2012},
and we used ANOVA to partition variation in the fitted values 
into contributions from midges, time, and distance. 
We then projected the observed data onto the PC axes.
The total contribution of each predictor to the observed community variation
was calculated as the sum of the axis loadings weighted by 
the corresponding variance contributions.
These two approaches provided statistical inference and visualization 
of community abundance and composition analogous to conventional ordination 
while appropriately accounting for temporal autocorrelation 
and maintaining an explicit link to taxon-specific responses. 



We fit the model with a Bayesian approach using Stan,
via the \texttt{rstan} \citep{Stan2018} package in R 4.0.3 \citep{R2020}.
We used 4 chains with 4000 iterations each (including 2000 iterations of "warm-up")
and assessed convergence based on effective sample size across the Markov chains, number of divergent
transitions, and potential scale reduction factor.
We used Gaussian priors with mean 0 and standard deviation 1 for the fixed effects,
Gamma priors with shape 1.5 and rate 3 for the standard deviations,
and a Gaussian prior with mean 0 and standard deviation 0.5 for the AR parameters.
The latter was truncated with a lower bound of zero to avoid quasi-cyclic dynamics,
but did not have an explicit upper bound. 
We used posterior medians as point estimates and 68\% quantiles as bounds of uncertainty intervals,
matching the coverage of standard errors (hereafter ``posterior standard errors'').
Data and code can be found at \url{https://github.com/lucasnell/trans\_trends}
and \url{https://github.com/lucasnell/trans\_trends\_pkg}.