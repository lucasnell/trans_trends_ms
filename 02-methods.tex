\section*{Methods}

Lake M\'{y}vatn has a tundra-subarctic climate, with a surrounding landscape
dominated by heathland and grassland vegetation \citep{Einarsson2004}.
Arthropod samples were collected every 15$\pm$4d from five sites around the
shoreline of M\'{y}vatn from May to August in 2008 through 2019.
Each site included 3--4 plots at various distances
(5m, 50m, 150m, and 500m) along a transect perpendicular to the lakeshore.
Midge deposition was estimated at each site using aerial infall traps \citep{Dreyer2015},
and activity-density of ground-dwelling arthropods was sampled using pitfall traps
\citep{Southwood2009}.
Activity-density is a measurement of relative abundance, which represents a combination
of both population size and behavioral phenomena, both of which potentially respond
to the allochthonous inputs of resources \citep{Ostfeld2000}.
For this analysis, we used the cumulative annual catch of predatory and
detritivorous arthropods (Figure \ref{fig:obs-data}),
including ground beetles (Carabidae), rove beetles (Staphylinidae),
harvestmen (Opiliones), ground spiders (Gnaphosidae),
sheet-weaving spiders (Linyphiidae), and wolf spiders (Lycosidae).



We analyzed taxon-level variation across sites and years using an extension of
linear mixed models.
Our model included (1) fixed effects for predictor variables (number of years since initial
sampling year, distance from the lake, and midges);
(2) random intercepts grouped by taxon, site, and plot; and
(3) random slopes for the predictor variables grouped by taxon
(see appendix for full mathematical description).
The fixed effects give the average response across taxa to the predictor variables,
while the corresponding random effects give the deviation for each taxon from this
average response \citep{Jackson2012}.
The overall response of a given taxon is the sum of the corresponding fixed and
random components.
Quantifying responses for each taxon in a single model 
allows for "shrinkage" or “partial pooling” of the taxon-specific responses towards the mean response,
which reduces the noisiness of individual estimates and ameliorates concerns of multiple
comparisons \citep{Gelman2012} that have been raised for the examination of
responses for many taxa separately \citep{Mcgarigal2013}.
We log-transformed cumulative midges catch and distance and then z-scored (subtracted the mean 
and divided by the standard deviation) all predictors
(excluding the intercept).



To formulate the autoregressive component of the mixed model, we defined a vector $\mathbf{y}$
consisting of transformed counts for each taxon in each plot through time.
Because population processes are generally multiplicative, we log-transformed the counts,
adding one to all values to accommodate zeros.
We then z-scored the
transformed counts within each taxon to give the responses of different taxa on
comparable scales \citep{Jackson2012}.
The rows of $\mathbf{y}$ were grouped as individual time series
(i.e., taxon--plot combinations), with observations within each group ordered by time.
From this vector, we constructed a lagged vector $\mathbf{y\sp{\prime}}$,
where the first element for each time series was zero followed by the first through
penultimate observations for that time series \citep{Ives2006}.
This allowed us to specify an autoregressive model for the $i$th observation of
$\mathbf{y}$ as
%
\begin{equation} \label{eq:y-i}
\begin{split}
    y_i &= \mathbf{x}_i^\text{T} {\boldsymbol\beta}_i +
        \left( \phi_{\text{taxon}[i]} \right)^{\text{time}_i - \text{time}\sp{\prime}_i}
        y\sp{\prime}_i + \varepsilon_i \\
    \varepsilon_i &\sim \mathcal{N} \left(0, \; \sigma_{\text{residual}} \right)
    \text{,}
\end{split}
\end{equation}
%
\noindent where $\mathbf{x}_i^\text{T}$ is the transposed vector
of predictor values for the $i$th observation,
${\boldsymbol\beta}_i$ is a vector of coefficients 
modeled hierarchically according to the random effects structure described above,
$\phi_{\text{taxon}[i]}$ is the taxon-specific autoregressive (AR) parameter,
$\text{time}_i$ is the current time (in years),
$\text{time}\sp{\prime}_i$ is the lagged time (defined as for  $y\sp{\prime}$),
and $\varepsilon_i$  is the Gaussian residual
error with standard deviation $\sigma_{\text{residual}}$.
Because $y_i$ consists of log-transformed abundances, equation \ref{eq:y-i} is
essentially a log-linear model of population dynamics, with growth rate
$\mathbf{x}_i^\text{T} {\boldsymbol\beta}_i$ and
density-dependence $\phi_{\text{taxon}[i]}$
where values closer to zero indicate stronger regulation \citep{Ives2010, Ziebarth2010}.
Exponentiating the elapsed time with base $\phi_{\text{taxon}[i]}$ allows the model
to accommodate unequal time steps \citep{Zuur2009}.
The primary difference between the present approach and more traditional linear models
with AR structures \citep[e.g.][]{Zuur2009} is the estimation of separate values of the AR
parameter for different groups (i.e. taxa).
This is an important extension, as it allows different members of a community with
different levels of temporal autocorrelation (caused by differences in the underlying
dynamics) to be included in a single model.
We parameterized the model with the predictors giving the change in $y_i$,
rather than the mean of the stationary distribution as is often done
for autoregressive models \citep{Harvey1990, Ives2006},
because this reduced the correlation between the estimates of temporal trends and AR parameters.



Community analyses often utilize dimension reduction to characterize 
variation in community composition \citep{Mcgarigal2013}.
However, with our approach it is possible to infer variation in community composition 
directly from the variation in taxon-specific responses.
We illustrate this in two ways.
First, we compared the full model to a series of simplified models that either
(1) excluded the taxon-specific deviations from the mean
response (i.e. random effects) or (2) excluded the overall response across the
full community (i.e. fixed and random effects).
The former provided inference for variation in community composition per se,
while the latter provided inference for variation in both composition and overall abundance.
We made these comparisons using the Leave-one-out information criterion (LOOIC) \citep{Vehtari2017},
which is a Bayesian analogue to the Akaike information criterion (AIC) 
and can be similarly interpreted in units of deviance.
Second, we used principal components analysis on fitted values from the model
to generate orthogonal axes
characterizing the expected variation in the community due to the predictors
\citep[similar to][]{Jackson2012},
and we used ANOVA to partition variation in the fitted values 
into contributions from midges, time, and distance. 
We then projected the observed data onto these PC axes.
The total contribution of each predictor to the observed community variation
was calculated as the sum of the axis loadings weighted by 
the corresponding variance contributions.
These two approaches provided formal inference and visualization of community composition analogous 
to conventional ordination while appropriately accounting 
for temporal autocorrelation and maintaining an explicit link to taxon-specific responses. 



We fit the models with a Bayesian approach using Stan \citep{Carpenter2017},
via the \texttt{rstan} \citep{Stan2018} package in R.
We used 6 chains with 4000 iterations each (including 2000 iterations of "warm-up")
and assessed convergence based on effective sample size of the Markov chain, number of divergent
transitions, and potential scale reduction factor.
We used Gaussian priors with mean 0 and standard deviation 1 for the fixed effects,
Gamma priors with shape 1.5 and rate 3 for the standard deviations,
and Gaussian priors with mean 0 and standard deviation 0.5 for the AR parameter.
The latter was truncated with a lower bound of zero to avoid quasi-cyclic dynamics,
but did not have an explicit upper bound. 
Nonetheless, the estimated values of the AR parameters were less than one, indicating stationarity. 