\documentclass[12pt]{article}
% \usepackage[sc]{mathpazo} % Like Palatino with extensive math support
\usepackage[letterpaper, margin=1in]{geometry}
% \usepackage{mathptmx} % Like Times New Roman
\usepackage{newtxtext,newtxmath}
\usepackage{fullpage}
\usepackage[authoryear,sectionbib,sort]{natbib}

% The following works for line spacing for ESA journals
% (per http://esapubs.org/esapubs/latexTIPsESA.pdf):
\linespread{1.9}

\usepackage[utf8]{inputenc}
\usepackage{lineno}
\usepackage{titlesec}
\titleformat{\section}[block]{\Large\bfseries\filcenter}{\thesection}{1em}{}
\titleformat{\subsection}[block]{\Large\itshape\filcenter}{\thesubsection}{1em}{}
\titleformat{\subsubsection}[block]{\large\itshape}{\thesubsubsection}{1em}{}
\titleformat{\paragraph}[runin]{\itshape}{\theparagraph}{1em}{}[. ]\renewcommand{\refname}{Literature Cited}

% For Icelandic ð symbol:
\DeclareTextSymbolDefault{\dh}{T1}
% Increased spacing in math mode:
\medmuskip=8mu % by default it is equal to 4 mu
\thickmuskip=10mu % by default it is equal to 5 mu

% Figures
\usepackage{graphicx}
\graphicspath{ {./figures/} }
% Table
\usepackage{booktabs}


%%%%%%%%%%%%%%%%%%%%%
% Line numbering
%%%%%%%%%%%%%%%%%%%%%
%
% Please use line numbering with your initial submission and
% subsequent revisions. After acceptance, please turn line numbering
% off by adding percent signs to the lines %\usepackage{lineno} and
% to %\linenumbers{} and %\modulolinenumbers[3] below.
%
% To avoid line numbering being thrown off around math environments,
% the math environments have to be wrapped using
% \begin{linenomath*} and \end{linenomath*}
%
% (Thanks to Vlastimil Krivan for pointing this out to us!)

\title{Quantifying community responses to environmental variation from replicate
time series}



% This version of the LaTeX template was last updated on
% November 8, 2019.

%%%%%%%%%%%%%%%%%%%%%
% Authorship
%%%%%%%%%%%%%%%%%%%%%
% Please remove authorship information while your paper is under review,
% unless you wish to waive your anonymity under double-blind review. You
% will need to add this information back in to your final files after
% acceptance.

\author{
Joseph S. Phillips$^{1,2,3,\dagger}$ \\
Lucas A. Nell$^{1,4,\dagger}$ \\
Jamieson C. Botsch$^{1}$}



\usepackage{amsmath} % for split math environment


\date{}

% so that no commas are used in citations:
\bibpunct{(}{)}{,}{a}{}{,}


\begin{document}

\raggedright
\setlength\parindent{0.25in}

\maketitle


\noindent{} 1. Department of Integrative Biology, University of Wisconsin, Madison, Wisconsin 53706 USA

\noindent{} 2. Department of Aquaculture and Fish Biology, H\'{o}lar University, Skagafj\"{o}r{\dh}ur 551 Iceland

\noindent{} 3. E-mail: joseph@holar.is

\noindent{} 4. E-mail: lucas@lucasnell.com

\noindent{} $\dagger$ Both authors contributed equally.



\bigskip

Running head: {Community responses from time series}

% \textit{Manuscript elements}: %Figure~1, figure~2, table~1, online appendices~A and B (including $-- figure~A1 and figure~A2). Figure~2 is to print in color.

\linenumbers{}
% \modulolinenumbers[3]

\clearpage

% ---------------------------------------------------------------------------------------
% ---------------------------------------------------------------------------------------
% Abstract
% ---------------------------------------------------------------------------------------
% ---------------------------------------------------------------------------------------


\section*{Abstract}

Times-series data for ecological communities are increasingly available from
medium- and long-term studies designed to track responses to environmental change.
However, classical multivariate methods for analyzing community composition are
generally inappropriate for time series, as they do not account for temporal
autocorrelation in abundances of community members.
Furthermore, these traditional approaches often obscure the connections between
responses at the community level versus those for individual taxa, limiting
their capacity to infer the mechanisms of community change.
We show how linear mixed models with group-specific temporal autocorrelation can be
used to infer both taxon- and community-level responses to predictor variables from
replicated time series data.
Variation in taxon-specific responses to environmental predictors is modeled using
random effects, which can then be used to characterize variation in community composition.
Moreover, the degree of autocorrelation is estimated separately for each taxon as
this is likely to vary depending on the underlying population dynamics.
We illustrate the utility of the approach by analyzing the response of a predatory
arthropod community to spatiotemporal variation in allochthonous resources in a
tundra landscape.
Our results show how mixed models with temporal autocorrelation provide a unified
approach to characterizing taxon- and community-level responses to environmental
variation through time.


\bigskip

\textit{Keywords}: {allochthonous subsidies; autoregressive model; linear mixed model;
M\'{y}vatn; time-series analysis; tundra arthropods
}





% ---------------------------------------------------------------------------------------
% ---------------------------------------------------------------------------------------
% Introduction
% ---------------------------------------------------------------------------------------
% ---------------------------------------------------------------------------------------

\section*{Introduction}

A central goal of ecology is to understand how environmental variation affects
population and community dynamics.
To this end, 
time-series data of ecological communities are increasingly available from monitoring programs such as the
National Ecological Observatory Network \citep{Keller2008} and
long-term field experiments \citep[e.g.,][]{Fraser2013, Cowles2018}.
Data of this sort provide great promise for answering long-standing questions of
how environmental factors drive community dynamics.
The responses of ecological communities to environmental predictors are often analyzed
using ordination methods (e.g., NMDS, RDA), which map variation in abundance or
occurrence onto orthogonal axes that provide synoptic assessments of community
variation \citep{Mcgarigal2013}.
However, when applied to community time series these approaches do not account for temporal
autocorrelation in the abundances of the populations that compose the community.
This is an important limitation, as failing to account for temporal autocorrelation
can lead to erroneous statistical inferences \citep{Ives2006}.
Furthermore, populations may vary in their degree of temporal autocorrelation,
due to differences in life histories or strengths of population regulation \citep{Ziebarth2010},
potentially exacerbating the challenges for statistical inference.
One approach to this problem is to use dimension reduction 
to produce composite community metrics for more traditional univariate time-series analyses \citep{Simpson2009}.
Unfortunately, this approach sacrifices valuable inferences about taxon-specific responses 
to environmental predictors.
Alternative approaches are designed to explicitly quantify dynamic interspecific interactions
but require relatively complex models and long time series \citep{Ives2003, Hampton2013}.
In contrast, community time series are often relatively short but contain
replication through space or across experimental units with the goal of
inferring the responses of communities to external drivers or experimental manipulations.

Here, we show how autoregressive mixed models (linear mixed models with temporal
autocorrelation structures) that account for observation error
can be used to infer taxon- and community-level responses
to predictor variables from replicated time series data.
The approach is based on the method of \cite{Jackson2012}, who model variation in
taxon-specific responses to predictors as random effects.
These random effects can in turn be used for statistical inferences on community composition,
which fundamentally derives from variation among taxa,
making explicit the link between the two organizational scales.
We extend this approach for time series by formulating models with separate
temporal autocorrelation structures for different taxonomic groups, accounting for
the fact that different taxa are likely to be characterized by different dynamics.
The degree of temporal autocorrelation itself provides information 
about the community dynamics \citep{Ives2003},
and its inclusion allows for valid statistical inferences on the effects of predictors 
when observations are temporally replicated \citep{Ives2006}.

We illustrate the utility of this method by analyzing the responses of 
terrestrial arthropods to spatiotemporal variation in allochthonous resources at Lake Mývatn, Iceland \citep{Einarsson2004}.
Mývatn has large emergences of midges (Diptera: Chironomidae) that emerge from the lake as adults 
and subsequently subsidize the terrestrial
plant \citep{Gratton2008, Krowiak2017} and arthropod \citep{Dreyer2012, Sanchez2018} communities.
The midges have large interannual fluctuations and decline
in deposition with distance from the lakeshore \citep{Dreyer2015}.
We apply autoregressive mixed models to assess the community response to the
highly variable midge subsidy, 
while simultaneously accounting for linear trends across time and distance from the lake.
Including those two covariates was important for isolating the effects of midges from those
of other variables that vary with distance (e.g., plant composition) or 
time (e.g., climate) \citep{Bowden2018}.
This study shows how autoregressive mixed models provide a statistically sound, 
easy-to-interpret method for characterizing community responses to
environmental variation in time and space.







% ---------------------------------------------------------------------------------------
% ---------------------------------------------------------------------------------------
% Methods
% ---------------------------------------------------------------------------------------
% ---------------------------------------------------------------------------------------




\section*{Methods}

Lake M\'{y}vatn has a tundra-subarctic climate, surrounded by a landscape
dominated by heathland and grassland vegetation \citep{Einarsson2004}.
Arthropod samples were collected every 15$\pm$4d from five sites around the
shoreline of M\'{y}vatn from May to August in 2008 through 2019.
Each site included 3--4 plots at various distances
(5m, 50m, 150m, and 500m) along a transect perpendicular to the lakeshore.
Midge deposition was estimated at each site using aerial infall cups \citep{Dreyer2015},
and activity-density of ground-dwelling arthropods was sampled using pitfall traps
\citep{Southwood2009}.
Activity-density is a measurement of relative abundance, which represents a combination
of both population size and behavioral phenomena, both of which are likely to respond
to the allochthonous inputs of resources \citep{Ostfeld2000}.
For this analysis, we used the cumulative annual catch of predatory and
detritivorous arthropods (Figure \ref{fig:obs-data}),
including ground beetles (Carabidae), rove beetles (Staphylinidae),
harvestmen (Opiliones), ground spiders (Gnaphosidae),
sheet-weaving spiders (Linyphiidae), and wolf spiders (Lycosidae).



We analyzed taxon-level variation across sites and years using an extension of
linear mixed models.
Our model included (1) fixed effects for predictor variables (time since initial
sampling year, distance from the lake, and midges);
(2) random intercepts grouped by taxon, site, and plot; and
(3) random slopes for the predictor variables grouped by taxon.
The fixed effects give the average response across taxa to the predictor variables,
while the corresponding random effects give the deviation for each taxon from this
average response \citep{Jackson2012}.
The overall response of a given taxon is the sum of the corresponding fixed and
random components.
Characterizing the responses for each taxon in a single model using random effects
allows for “partial pooling” of the taxon-specific responses towards the mean response,
which reduces the noisiness of individual estimates and ameliorates concerns of multiple
comparisons \citep{Gelman2012} that have been raised for the examination of
responses for many taxa separately \citep{Mcgarigal2013}.



To formulate the autoregressive mixed model, we defined a vector $\mathbf{y}$
consisting of the transformed counts for each taxon in each plot through time.
Because population processes are generally multiplicative, we log-transformed the counts,
adding one to all values to accommodate zeros.
We then z-scored (subtracted the mean and divided by the standard deviation) the
transformed counts within each taxon to give the responses of different taxa on
comparable scales \citep{Jackson2012}.
The rows of $\mathbf{y}$ were grouped as individual time series
(i.e. taxon--plot combinations), with observations within each group ordered by time.
From this vector, we constructed a lagged vector $\mathbf{y\sp{\prime}}$,
where the first element for each time series was zero followed by the first through
penultimate observations for that time series \citep{Ives2006}.
This allowed us to specify an autoregressive model for the $i$th observation of
$\mathbf{y}$ as
%
\begin{equation} \label{eq:y-i}
\begin{split}
    y_i &= \mathbf{x}_i^\text{T} {\boldsymbol\beta}_i +
        \left[ \phi_{\text{taxon}[i]} \right]^{\text{time}_i - \text{time}\sp{\prime}_i}
        y\sp{\prime}_i + \varepsilon_i \\
    \varepsilon_i &\sim \mathcal{N} \left(0, \; \sigma_{\text{residual}} \right)
    \text{,}
\end{split}
\end{equation}
%
\noindent where $\mathbf{x}_i^\text{T}$ is the transposed vector
of predictor values for the $i$th observation,
${\boldsymbol\beta}_i$ is a vector of coefficients,
$\phi_{\text{taxon}[i]}$ is the taxon-specific autoregressive (AR) parameter,
$\text{time}_i$ is the current time (in years),
$\text{time}\sp{\prime}_i$ is the lagged time (defined as for  $y\sp{\prime}$),
and $\varepsilon_i$  is the Gaussian residual
error with standard deviation $\sigma_{\text{residual}}$.
Because $y_i$ consists of log-transformed abundances, equation \ref{eq:y-i} is
essentially a log-linear model of population dynamics, with growth rate
$\mathbf{x}_i^\text{T} {\boldsymbol\beta}_i$ and
density-dependence $\phi_{\text{taxon}[i]}$
where values closer to zero indicate stronger regulation \citep{Ives2010, Ziebarth2010}.
The exponentiation of $\phi_{\text{taxon}[i]}$ by the change in time allows the model
to accommodate unequal time steps \citep{Zuur2009}.
The primary difference between the present approach and more traditional linear models
with AR structures \citep[e.g.][]{Zuur2009} is the estimation of separate values of the AR
parameter for different groups (i.e. taxa).
This is an important extension, as it allows different members of a community with
different levels of temporal autocorrelation (caused by differences in the underlying
dynamics) to be included in a single model.
We parameterized the model with the predictors giving the change in $y_i$,
rather than the mean of the stationary distribution as is often done
for autoregressive models \citep{Harvey1990, Ives2006},
because this reduced the correlation between the estimates of temporal trends
(contained in ${\boldsymbol\beta}_i$; see below) and AR parameters.


For our example application,
we were interested in characterizing the response of the predatory arthropod
community to midge deposition, while also accounting for potential trends through
time and space due to other factors.
Therefore, we defined the transposed vector of predictor values for the $i$th observation
of $\mathbf{y}$ as
%
\begin{equation} \label{eq:x-vec}
    \mathbf{x}_i^\text{T} = \begin{bmatrix}
        1 &
        {}^z\hspace*{-1pt}\text{time}_i &
        {}^z\hspace*{-1pt}\text{dist}_i &
        {}^z\hspace*{-1pt}\text{midge}_i
    \end{bmatrix}\text{,}
\end{equation}
%
\noindent with 1 corresponding to the intercept (hereafter denoted ``int'').
We log-transformed midges and distance and then z-scored all predictors
(excluding the intercept).
We defined a vector of corresponding coefficients defined as
%
\begin{equation} \label{eq:beta-vec}
{\boldsymbol\beta}_i = \begin{bmatrix}
    {}^\text{int}\hspace*{-1pt}\beta_i \\
    {}^\text{time}\hspace*{-1pt}\beta_i \\
    {}^\text{dist}\hspace*{-1pt}\beta_i \\
    {}^\text{midge}\hspace*{-1pt}\beta_i
    \end{bmatrix}\text{.}
\end{equation}
%
\noindent We modeled the coefficients hierarchically \citep[following][]{Jackson2012} as
%
\begin{equation} \label{eq:betas}
\begin{split}
    {}^\text{int}\hspace*{-2pt}\beta_i &= {}^\text{int}\hspace*{-1pt}\alpha +
        {}^\text{int}\hspace*{-1pt}\zeta_{\text{taxon}[i]} +
        {}^\text{int}\hspace*{-1pt}\zeta_{\text{site}[i]} +
        {}^\text{int}\hspace*{-1pt}\zeta_{\text{plot}[i]} \\
    {}^\text{time}\hspace*{-1pt}\beta_i &= {}^\text{time}\hspace*{-1pt}\alpha +
            {}^\text{time}\hspace*{-1pt}\zeta_{\text{taxon}[i]} \\
    {}^\text{dist}\hspace*{-1pt}\beta_i &= {}^\text{dist}\hspace*{-1pt}\alpha +
            {}^\text{dist}\hspace*{-1pt}\zeta_{\text{taxon}[i]} \\
    {}^\text{midge}\hspace*{-1pt}\beta_i &= {}^\text{midge}\hspace*{-1pt}\alpha +
            {}^\text{midge}\hspace*{-1pt}\zeta_{\text{taxon}[i]}
\end{split}
\end{equation}
%
\noindent with fixed effects ${}^k\hspace*{-1pt}\alpha$ and random effects
${}^k\hspace*{-1pt}\zeta_{\text{g}[i]}$ for predictor $k$ (e.g. midge)
and the level of grouping variable $g$ (e.g. taxon) associated  with the $i$th observation.
The random effects were modeled as Gaussian distributions with
standard error $\sigma_g$:
%
\begin{equation} \label{eq:zetas}
    {}^k\hspace*{-1pt}\zeta_{g[i]} \sim
        \mathcal{N}\left(0, \; {}^k\hspace*{-1pt}\sigma_g \right)
\end{equation}



Variation in community composition per se due to predictor variables arises from
variation in the taxon-specific responses, and the magnitude of this variation for
predictor $k$ is given by ${}^k\hspace*{-1pt}\sigma_\text{taxon}$.
Therefore, the model characterizes both taxon-level and community-level responses
to the predictors.
To make the community responses more explicit, we visualized variation in the
community along axes of variation most strongly associated with the predictor
variables of the model (midges, time, and distance).
The method we used is analogous to more conventional ordination approaches
\citep{Mcgarigal2013}, but makes explicit the connection to taxon-specific responses.
To start, we generated model-predicted values due to variation in the predictors,
using values evenly spaced across the observed range for each predictor.
We excluded the random intercepts grouped by transect and plot.
We then converted these predicted values into an “event-by-taxon” matrix
(analogous to a “site-by-species” matrix) \citep{Mcgarigal2013} and
performed a Principal Components Analysis (PCA) to generate orthogonal axes
characterizing the expected variation in the community due to the predictors
\citep[similar to][]{Jackson2012}.
Variation in the PC axes was partitioned into contributions from the predictors using
ANOVA.
Finally, we projected the observed data onto the PC axes, which allowed us to
characterize the extent to which the PC axes (and therefore the effects of the
predictors) accounted for variation in the observed data.
This approach includes both changes in composition per se (the typical focus of
ordination analyses) and the overall community response.
This is a strength, as it is difficult to assess the relevance of variation in
composition without the context of the overall response.
However, the method can be modified to only include changes in composition by
excluding the fixed effects (${}^k\hspace*{-1pt}\alpha$) from the generation of
model predictions.





% =======================================================================================
% =======================================================================================
% =======================================================================================
% =======================================================================================
% =======================================================================================
% =======================================================================================
% =======================================================================================



We fit the models in a Bayesian framework using Stan \citep{Carpenter2017}
implemented using the rstan \citep{Stan2018} package in R,
although this method could be readily adapted to other model fitting software
such as WinBUGS, JAGS, and AD Model Builder \citep{Bolker2013}.
We used Gaussian distributions with mean 0 and standard deviation 1 as
weakly-informative priors \citep{Gelman2017}
for fixed effects, random effects, and standard deviations,
with the latter truncated at 0 since standard deviations must be positive.
We used a similar prior for the AR parameter, but with a standard deviation 0.5 to
aid with convergence towards stationarity (i.e. $|\phi_{\text{taxon}[i]}|<1$) without
providing an explicit upper bound.
However, we did truncate the lower bound of the prior at 0 to prevent negative
values that would give quasi-cyclic behaviors, which is the convention when
including temporal autocorrelation in mixed models \citep{Zuur2009}.
Random effects were specified with a non-centered parameterization
to improve the efficiency of parameter space exploration \citep{Betancourt2015}.
We assessed convergence using the standard diagnostics provided by Stan,
including effective sample size of the Markov chain, number of divergent
transitions, and potential scale reduction factor.
We used posterior medians as point estimates,
and 16\% and 84\% quantiles
for the bounds of 68\% uncertainty intervals (UI68\%),
analogous to the coverage of standard errors.

After fitting the full model to the observed data, we compared it to reduced models
to evaluate the strength of evidence for community responses to midges, time,
and distance.
We specified the reduced models by excluding coefficients associated with the
predictors in two ways: (1) excluding the taxon-specific deviations from the mean
response (i.e. random effects) and (2) excluding the overall response across the
full community (i.e. fixed and random effects).
Because the goal was to provide inference on the full model, rather than identify
an ``optimal'' model, we only compared the full model to reduced models
excluding terms associated with a single predictor.
We compared models using an approximation to Leave-One-Out (LOO) cross validation,
based on the posterior distribution of log-likelihoods for each model
\citep{Vehtari2017}.
We report twice the difference in the ``estimated log pointwise predictive density''
between the full and reduced models (hereafter ``LOO deviance''), which is in units of
deviance and can interpreted in a manner similar to the difference in the
Akaike Information Criteria between model pairs (i.e. $\Delta$AIC) as
commonly used in frequentist settings.
Note that while LOO (and similar metrics such as AIC) does not account for the
sequential nature in which times-series data are observed and so does not formally
assess the capacity of a model for predicting future observations,
it is still useful for retrospectively determining
how well a model characterizes a given set of data.




% ---------------------------------------------------------------------------------------
% ---------------------------------------------------------------------------------------
% Results
% ---------------------------------------------------------------------------------------
% ---------------------------------------------------------------------------------------

\section*{Results}

The AR coefficients were generally similar across taxa and of modest magnitude,
indicating moderate levels of temporal autocorrelation across years
(Figure \ref{fig:coefs}B).
Ground spiders were an exception,
with an AR coefficient near zero indicating limited autocorrelation.
The relatively low autocorrelation is apparent from the time series,
where ground spiders appear more tightly constrained to their mean abundance
than was the case for other taxa (Figure \ref{fig:obs-data}A),
which results in less autocorrelated (i.e. ``faster'') fluctuations through time
\citep{Ziebarth2010}. 
While caution is warranted when interpreting dynamical features of short time series,
this result implies that ground spiders were more tightly constrained to their mean abundance
as determined by the environmental predictors, while the other taxa were only modestly constrained.
However, none of the AR coefficients were close to 1, 
implying that the populations were statistically stationary through time 
once accounting for the predictors, including linear time trends.



The overall community response to midge deposition was positive,
as indicated by both the coefficient estimates (Figure \ref{fig:coefs}A,C)
and the improvement of the full model
relative to the reduced model (Table \ref{tab:model-summary}: \emph{LOOIC}).
The response to midges was somewhat variable across taxa,
as judged by the taxon-specific coefficients (Figure \ref{fig:coefs}A)
and the posterior distribution
for the random effect standard deviation (Figure \ref{fig:coefs}D).
Specifically, ground spiders had the most positive response,
while sheet weavers had the weakest (largely neutral) response.  
However, this variation among taxa made a relatively small contribution
to the model fit (Table \ref{tab:model-summary}: \emph{LOOIC}).
In contrast to the response to midges, there was large variation among taxa in
linear trends through time and distance from the lake (Figure \ref{fig:coefs}A,D);
this taxon-specific variation dominated the contribution to the model fit.
Responses to distance were particularly variable, with wolf spiders 
increasing and rove beetles decreasing with distance from the lake.
Overall, the magnitudes of the taxon-specific responses to time and distance
were larger than the responses to midges.



The taxon-specific responses provided the basis for inferring variation 
in both community composition per se and overall community abundance.
Changes in composition arose when taxa differed in their responses
to predictors, while changes in overall abundance arose from consistent
responses among taxa.
For example, taxa varied in their responses to distance from the lake 
(Figure \ref{fig:coefs}; Table \ref{tab:model-summary}: \emph{LOOIC}),
which in turn implied that community composition varied with distance.
In the PCA, this resulted in variation in the taxon-response vectors
along PC2 (Figure \ref{fig:pca}A), 
which was most strongly associated with distance 
(Figure \ref{fig:pca}C; Table \ref{tab:model-summary}: \emph{Variance partitioning}).
Community composition also varied through time, 
which primarily manifested in variation along PC3 (Figure \ref{fig:pca-23}).
In contrast, the abundance of all taxa except for sheet weavers increased with midges,
resulting in an positive overall response of community abundance along PC1 (Figure \ref{fig:pca}A,D).
Distance accounted for most of the observed community variation,
followed by time, and midges (Table \ref{tab:model-summary}). 
This implies that (1) changes in composition per se
with distance and time caused most of the community variation, and 
(2) variation in overall abundance due to midges was more modest,
but still statistically meaningful.
Together, linear responses to midges, distance, and time accounted for 
68\% of the observed community variation.






% ---------------------------------------------------------------------------------------
% ---------------------------------------------------------------------------------------
% Discussion
% ---------------------------------------------------------------------------------------
% ---------------------------------------------------------------------------------------



\section*{Discussion}

Here, we show how linear mixed models with temporal autocorrelation can be used to quantify
community responses to environmental variation from replicated time series observations. 
Our approach extends the previous use of mixed models for estimating taxon- and community-level
responses to environmental variation \citep{Bartrons2015, Jackson2012} by
incorporating temporal autocorrelation with group-specific values for the autoregressive parameter. 
While our model formulation accounted for space using linear trends (due to the nature of the example data),
the method could be extended to explicitly include spatial autocorrelation when appropriate
\citep[similar to][]{Bartrons2015}.
The approach of modeling community responses at the taxon level
provides a natural way to accommodate the temporal autocorrelation
structure of community times series data,
because community dynamics are manifestations of the dynamics of constituent populations 
(regardless of whether those populations interact).
Furthermore, the taxon-specific autoregressive parameter
estimated by our approach has a clear population-dynamic interpretation as the rate at which the
population returns to its central tendency, which is closely tied to classical ecological
concepts of population regulation \citep{Nicholson1933, Ziebarth2010}. 
In our application to the M\'{y}vatn arthropod community,
we found substantial variation in the AR parameter across taxa.
This suggests that they either experience different dynamics or that those dynamics play out at different time scales.
However, caution is warranted when drawing such inferences from short time series
and when not explicitly accounting for interspecific interactions.



While our approach is formulated in terms of taxon-specific responses,
we illustrate two methods for how these can be extended to inferences about community composition.
The first entailed comparing the full model to reduced models that eliminated different aspects of 
taxon-level responses to the predictors.
The comparison of the full model to the reduced model omitting variation among taxon responses provided a statistical
assessment of whether composition per se varied across the predictors.
Comparison to the model omitting response to the predictors altogether provided an assessment of the overall community response.
The second approach applied principal components analysis to the fitted values from the model \citep[similar to][]{Jackson2012}, 
which allowed variation in community composition to be quantified in a manner analogous to traditional ordination
while appropriately accounting for temporal autocorrelation. 
We then partitioned the variance in the principal components analyses into contributions from the predictor variables.
This exemplifies the general dictum that it is better to `analyze, then aggregate' than it
is to `analyze the aggregate' \citep{Clark2011}.
In our example application to the M\'{y}vatn data,
we found that most of the community variation was due to changes in composition per se
resulting from heterogeneous taxon-specific responses to distance from the lake and across years.
However, most taxa were positively associated with midge deposition,
resulting in elevated community abundance overall.



This study fits into a larger effort to understand the role 
of allochthonous inputs on recipient communities \citep{Polis1997}.
One of the challenges in identifying the effects of allochthonous pulses 
from long-term observational data is that they are likely to be conflated
with other factors due to the manner in which they vary through space and time \citep{Yang2008}.
By using linear mixed effects models with group-specific temporal autocorrelation, 
we were able to partition the influence of three interrelated drivers of community composition 
to quantify the effect allochthonous resources had on a community.
We found a positive overall response of predator abundance to midge deposition, which
is consistent with previous studies at M\'{y}vatn \citep{Dreyer2012, Hoekman2011, Sanchez2018}
and in other systems \citep{Murphy2012, Ostfeld2000}.
Given the ubiquity of spatiotemporal associations between different drivers of ecological communities, 
our approach has potential utility across many subdisciplines of community ecology.









% ---------------------------------------------------------------------------------------
% ---------------------------------------------------------------------------------------
% Acknowledgments
% ---------------------------------------------------------------------------------------
% ---------------------------------------------------------------------------------------
% You may wish to remove the Acknowledgments section while your paper
% is under review (unless you wish to waive your anonymity under
% double-blind review) if the Acknowledgments reveal your identity.
% If you remove this section, you will need to add it back in to your
% final files after acceptance.

% \section*{Acknowledgments}
%
% OEC would like to thank the world. GHC is much indebted to the solar system. AQE was supported by a generous grant from the Milky Way (MW/01010/987654).


% ---------------------------------------------------------------------------------------
% ---------------------------------------------------------------------------------------
% Appendices
% ---------------------------------------------------------------------------------------
% ---------------------------------------------------------------------------------------

% \newpage{}
%
% \input{app_A}


% ---------------------------------------------------------------------------------------
% ---------------------------------------------------------------------------------------
% Literature Cited
% ---------------------------------------------------------------------------------------
% ---------------------------------------------------------------------------------------



\bibliographystyle{ecology.bst}
\clearpage

\bibliography{refs.bib}


\clearpage

\begin{table}
\caption{\label{tab:model-summary}
Model inference.
LOO deviance was calculated from the posterior distribution of log-likelihoods
and compares the fit of the full model to a reduced model excluding either
deviations in taxon-specific responses from the mean response or
the overall response (including mean and taxon-specific components).
The scale of LOO deviance is comparable to $\Delta$AIC.
The variance partitioning was calculated using ANOVA,
with the sum-of-squares for each component scaled by the total sum-of-squares.
Parenthetical values show relative loadings for the PC axes.
The `Total' column is calculated as the sum across PC axes for each row,
with each axis weighted by the contribution of each predictor when relevant.}
\begin{tabular}{rrrrrrrr}
\toprule
 & \multicolumn{2}{c}{LOO deviance} & & \multicolumn{4}{c}{Variance partitioning} \\
 \cmidrule{2-3} \cmidrule{5-8}
 & \multicolumn{1}{c}{Taxon-variation} & \multicolumn{1}{c}{Overall} & &
    \multicolumn{1}{c}{PC1} & \multicolumn{1}{c}{PC2} & \multicolumn{1}{c}{PC3} &
    \multicolumn{1}{c}{Total} \\
& \multicolumn{1}{c}{(random)} & \multicolumn{1}{c}{(fixed + random)} & &
    \multicolumn{1}{c}{(0.34)} & \multicolumn{1}{c}{(0.17)} &
    \multicolumn{1}{c}{(0.17)} & \multicolumn{1}{c}{(0.68)} \\
\midrule
time & 37 & 34 &  & 0.00 & 0.28 & 0.84 & (0.19)\\
distance & 64 & 64 &  & 0.99 & 0.04 & 0.01 & (0.34)\\
midges & 4 & 15 &  & 0.01 & 0.67 & 0.15 & (0.14)\\
\bottomrule
\end{tabular}
\end{table}


\clearpage

\begin{figure}
\centering
\includegraphics{fig1.pdf}
\caption{\label{fig:obs-data}
Community time series data for 6 predatory arthropod taxa at Lake M\'{y}vatn.
(A) Transformed abundance of arthropods through time.
Narrow gray lines are time series grouped by site and distance.
(B) Transformed abundance of arthropods versus distance from the lake.
Thick red lines are transformed midge abundances averaged by (A) year or (B) distance.
Thick blue lines are transformed abundances averaged by taxon and
(A) year or (B) distance.
Relative abundances were measured using activity-densities that were log-transformed,
then z-scored within each taxon.
}
\end{figure}




\clearpage

\begin{figure}
\centering
\includegraphics{fig2.pdf}
\caption{\label{fig:coefs}
Parameter estimates from the autoregressive mixed model, including
(A) taxon-specific responses to each predictor,
(B) autoregressive coefficients for each taxon, and
(C,D) posterior distributions for the fixed effects and random effect standard
deviations, respectively.
(A) The shaded regions correspond with the 68\% uncertainty interval
(analogous to the coverage of standard errors) for the mean response across taxa
(i.e. fixed effects shown in panel C).
(A,B) Points are posterior medians, and error bars are 68\% uncertainty intervals.
In all panels, gray vertical lines correspond to 0.
}
\end{figure}



\clearpage

\begin{figure}
\centering
\includegraphics{fig3.pdf}
\caption{\label{fig:pca}
Principal components analysis (PCA) of community variation, showing
(A) taxon-response vectors and (B--D) model predictors projected onto the PC axes.
The PCA is based on the taxon-specific responses to time, distance, and midges,
as inferred from the model.
Therefore, the PC axes are aligned to maximize variation associated with responses
to the predictor variables, similar to ordination methods such as ``redundancy analysis.''
The observed data were then projected onto these axes so that variation accounted for
by the model could be visualized in the context of the data.
The taxon vector overlays in panel A are scaled relative to vectors in
B--D for clarity of visualization.
}
\end{figure}


% % Figure S1
% \caption{\label{fig:pca-23}
% Principal components analysis (PCA) of community variation, projected onto
% PC2 and PC3, showing (A) taxon-response vectors and
% (B--D) model predictors projected onto the PC axes.
% The PCA is based on the taxon-specific responses to time, distance, and midges,
% as inferred from the model.
% Therefore, the PC axes are aligned to maximize variation associated with responses
% to the predictor variables, similar to ordination methods such as ``redundancy analysis.''
% The observed data were then projected onto these axes so that variation accounted for
% by the model could be visualized in the context of the data.
% The taxon vector overlays in panel A are scaled relative to vectors in
% B--D for clarity of visualization.
% }
%
% % Figure S2
% \caption{\label{fig:pca-13}
% Principal components analysis (PCA) of community variation, projected onto
% PC1 and PC3, showing (A) taxon-response vectors and
% (B--D) model predictors projected onto the PC axes.
% The PCA is based on the taxon-specific responses to time, distance, and midges,
% as inferred from the model.
% Therefore, the PC axes are aligned to maximize variation associated with responses
% to the predictor variables, similar to ordination methods such as ``redundancy analysis.''
% The observed data were then projected onto these axes so that variation accounted for
% by the model could be visualized in the context of the data.
% The taxon vector overlays in panel A are scaled relative to vectors in
% B--D for clarity of visualization.
% }


\end{document}
