
\section*{Abstract}

Times-series data for ecological communities are increasingly available from
medium- and long-term studies designed to track responses to environmental change.
However, classical multivariate methods for analyzing community composition are
generally inappropriate for time series, as they do not account for temporal
autocorrelation in abundances of community members.
Furthermore, these traditional approaches often obscure the connections between
responses at the community level versus those for individual taxa, limiting
their capacity to infer the mechanisms of community change.
We show how linear mixed models with group-specific temporal autocorrelation can be
used to infer both taxon- and community-level responses to predictor variables from
replicated time series data.
Variation in taxon-specific responses to environmental predictors is modeled using
random effects, which can then be used to characterize variation in community composition.
Moreover, the degree of autocorrelation is estimated separately for each taxon as
this is likely to vary depending on the underlying population dynamics.
We illustrate the utility of the approach by analyzing the response of a predatory
arthropod community to spatiotemporal variation in allochthonous resources in a
tundra landscape.
Our results show how mixed models with temporal autocorrelation provide a unified
approach to characterizing taxon- and community-level responses to environmental
variation through time.
