
\section*{Abstract}

Time-series data for ecological communities are increasingly available from
long-term studies designed to track species responses to environmental change.
However, classical multivariate methods for analyzing community composition
have limited applicability for time series, as they do not account for temporal
autocorrelation in community-member abundances.
Furthermore, traditional approaches often obscure the connections between
responses at the community level and those for individual taxa, limiting
their capacity to infer mechanisms of community change.
We show how linear mixed models that account for group-specific temporal autocorrelation 
and observation error can be used to infer both taxon- and community-level responses 
to environmental predictors from replicated time-series data.
Variation in taxon-specific responses to predictors is modeled using
random effects, which can be used to characterize variation in community composition.
Moreover, the degree of autocorrelation is estimated separately for each taxon, since
this is likely to vary due to differences in their underlying population dynamics.
We illustrate the utility of the approach by analyzing the response of a predatory
arthropod community to spatiotemporal variation in allochthonous resources in a
subarctic landscape.
Our results show how mixed models with temporal autocorrelation provide a unified
approach to characterizing taxon- and community-level responses to environmental
variation through time.