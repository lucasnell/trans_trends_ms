
\section*{Discussion}

Here, we show how linear mixed models with temporal autocorrelation can be used to quantify
community responses to environmental variation from replicated time series observations. 
Our approach extends the previous use of mixed models for estimating taxon- and community-level
responses to environmental variation \citep{Bartrons2015, Jackson2012} by
incorporating temporal autocorrelation with group-specific values for the autoregressive parameter. 
While our model formulation accounted for space using linear trends (due to the nature of the example data),
the method could be extended to explicitly include spatial autocorrelation when appropriate
\citep[similar to][]{Bartrons2015}.
The approach of modeling community responses at the taxon level
provides a natural way to accommodate the temporal autocorrelation
structure of community times series data,
because community dynamics are manifestations of the dynamics of constituent populations 
(regardless of whether those populations interact).
Furthermore, the taxon-specific autoregressive parameter
estimated by our approach has a clear population-dynamic interpretation as the rate at which the
population returns to its central tendency, which is closely tied to classical ecological
concepts of population regulation \citep{Nicholson1933, Ziebarth2010}. 
In our application to the M\'{y}vatn arthropod community,
we found substantial variation in the AR parameter across taxa.
This suggests that they either experience different dynamics or that those dynamics play out at different time scales.
However, caution is warranted when drawing such inferences from short time series
and when not explicitly accounting for interspecific interactions.



While our approach is formulated in terms of taxon-specific responses,
we illustrate two methods for how these can be extended to inferences about community composition.
The first entailed comparing the full model to reduced models that eliminated different aspects of 
taxon-level responses to the predictors.
The comparison of the full model to the reduced model omitting variation among taxon responses provided a statistical
assessment of whether composition per se varied across the predictors.
Comparison to the model omitting response to the predictors altogether provided an assessment of the overall community response.
The second approach applied principal components analysis to the fitted values from the model \citep[similar to][]{Jackson2012}, 
which allowed variation in community composition to be quantified in a manner analogous to traditional ordination
while appropriately accounting for temporal autocorrelation. 
We then partitioned the variance in the principal components analyses into contributions from the predictor variables.
This exemplifies the general dictum that it is better to `analyze, then aggregate' than it
is to `analyze the aggregate' \citep{Clark2011}.
In our example application to the M\'{y}vatn data,
we found that most of the community variation was due to changes in composition per se
resulting from heterogeneous taxon-specific responses to distance from the lake and across years.
However, most taxa were positively associated with midge deposition,
resulting in elevated community abundance overall.



This study fits into a larger effort to understand the role 
of allochthonous inputs on recipient communities \citep{Polis1997}.
One of the challenges in identifying the effects of allochthonous pulses 
from long-term observational data is that they are likely to be conflated
with other factors due to the manner in which they vary through space and time \citep{Yang2008}.
By using linear mixed effects models with group-specific temporal autocorrelation, 
we were able to partition the influence of three interrelated drivers of community composition 
to quantify the effect allochthonous resources had on a community.
We found a positive overall response of predator abundance to midge deposition, which
is consistent with previous studies at M\'{y}vatn \citep{Dreyer2012, Hoekman2011, Sanchez2018}
and in other systems \citep{Murphy2012, Ostfeld2000}.
Given the ubiquity of spatiotemporal associations between different drivers of ecological communities, 
our approach has potential utility across many subdisciplines of community ecology.

