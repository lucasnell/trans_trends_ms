

\section*{Discussion}

Here, we show how linear mixed models with temporal autocorrelation can be used to
quantify community responses to environmental variation from replicated time series
observations.
Our approach extends the previous use of mixed models for estimating taxon- and
community-level responses to environmental variation through space
\citep{Jackson2012, Bartrons2015}, by incorporating temporal autocorrelation with
group-specific values for the autoregressive parameter.
This is an important extension, as taxa populations with different natural histories
likely have different degrees of autocorrelation in their population dynamics.
Furthermore, we show how predicted values from the model estimates can be combined
with principal components analysis \citep[following][]{Jackson2012} to visualize
community composition along axes of variation most strongly associated with
environmental variation.
This analysis makes explicit the connections between taxon- and community-level
variation and provides a conceptual link between the mixed model approach and
conventional ordination methods.


A central feature of our approach is its formulation in terms of taxon-specific
responses to environmental drivers, rather than aggregate measures of community
composition.
This stands in contrast to the previous approach of using some dimension-reduction
technique (e.g. PCA) to generate indices of community variation to use as
response variables in subsequent time-series analyses \citep{Simpson2009}.
In such analyses, the temporal autocorrelation structure of the aggregate measure
is a complex amalgamation of processes at lower levels of resolution.
This may be reasonable when the aggregate measure is of interest in and of itself,
for example when being used as a bioindicator of ecosystem status \citep{Bennion2015}.
However, community dynamics are manifestations of the dynamics of constituent populations.
Therefore, it will often be valuable to model the data in a fashion that preserves the
autocorrelation structure of those populations the community comprises.
The taxon-specific autoregressive parameter estimated by our approach has a
clear population-dynamic interpretation as the rate at which the population returns
to its central tendency, which in turn is closely tied to classical ecological
concepts of population regulation \citep{Nicholson1933, Ziebarth2010}.
This is a special case of a more general argument in favor of quantifying
taxon-specific variation rather than aggregate measures of composition,
following the motto that it is better to ‘analyze, then aggregate’ than it is to
‘aggregate, then analyze’ \citep{Clark2011}.



To illustrate its utility, we used our method to quantify the response of a predatory
arthropod community to spatiotemporal variation in allochthonous resources at
Lake M\'{y}vatn in northern Iceland.
We found a positive overall response of predator abundance to midge deposition,
which is consistent with previous studies at M\'{y}vatn
\citep{Hoekman2011, Hoekman2019, Dreyer2012, Sanchez2018} and
expected from studies of allochthonous subsidies in other systems
\citep{Murphy2012, Ostfeld2000, Yang2010}.
However, the variation among taxa in their responses to midges was fairly limited,
which means that variation in midge deposition was primarily associated with changes
in overall abundance in the community, rather than composition per se.
In contrast, trends in taxon abundance varied substantially across time and space,
which manifested as spatiotemporal variation in community composition.
Many of the factors determining the strength of pulsed allochthonous resources on
recipient communities \citep[e.g. landscape features;][]{Polis1997} also directly
affect the composition of the communities receiving those allochthonous inputs.
The ability of our modeling approach to partition variance between the spatiotemporal
patterns that affect both the community composition and the size of the allochthonous
inputs on the abundance of community members allowed for important insights into the
community dynamics of predators around Lake M\'{y}vatn.
Furthermore, this approach allowed for the utilization of long-term observational
data to gather the insights about the predatory arthropod community to natural
variation in allochthonous inputs, which represents an important contribution to the
allochthonous resource pulse literature \citep{Yang2008}.
Covariation between different drivers of ecological communities is common,
therefore this approach may have utility across many subdisciplines of community ecology.






