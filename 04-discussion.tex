
\section*{Discussion}

Here, we show how linear mixed models with temporal autocorrelation can be used to quantify
community responses to environmental variation from replicated time series. 
The approach extends the previous use of mixed models for estimating taxon- and community-level
responses to environmental variation \citep{Jackson2012, Bartrons2015} by
incorporating temporal autocorrelation with group-specific values for the autoregressive parameter. 
While our model formulation accounted for space using linear trends (due to the nature of the example data),
the method could be extended to explicitly include spatial autocorrelation when appropriate
\citep[similar to][]{Bartrons2015}.
The approach of modeling community responses at the taxon level
provides a natural way to accommodate the temporal autocorrelation
structure of community times-series data,
because community dynamics are manifestations of the dynamics of constituent populations.
Furthermore, the taxon-specific autoregressive parameter
estimated by our approach has a clear population-dynamic interpretation as the
strength with which the population is drawn towards its central tendency, 
which is closely tied to classical ecological
concepts of population regulation \citep{Nicholson1933, Ziebarth2010}. 
In our application to the M\'{y}vatn arthropod community,
the AR parameters for all taxa were well below 1,
indicating that the populations were statistically stationary through time 
after accounting for linear time trends and responses to midge deposition.
Taxa also varied substantially in their AR parameters,
with the dynamics of ground spiders appearing to be more tightly constrained than for the other taxa.
However, caution is warranted when drawing such inferences from short time series
and when not explicitly accounting for interspecific interactions.



While our approach is formulated in terms of taxon-specific responses,
we illustrate two methods for how these can be extended to draw inferences about community composition.
First, the model provides estimates of both the mean response (fixed effects) 
and the variation in responses (random effects) among taxa.
When posterior distributions for the means or standard deviations are concentrated away from zero,
this indicates substantial variation in community-wide abundance and composition per se, respectively. 
Second, principal components analysis can be applied to the fitted values from the model \citep[similar to][]{Jackson2012}, 
which allows variation in community composition to be quantified in a manner analogous to traditional ordination
while appropriately accounting for temporal autocorrelation. 
The variance in the principal components analyses can then be partitioned into contributions from the predictor variables.
This exemplifies the general dictum that it is better to `analyze, then aggregate' than it
is to `analyze the aggregate' \citep{Clark2011}.
In our example application to the M\'{y}vatn data,
most of the community variation was due to changes in composition per se
resulting from heterogeneous taxon-specific responses to distance from the lake and across time.
However, most taxa were positively associated with midge deposition,
resulting in elevated abundance across the entire predatory arthropod community.



This study fits into a larger effort to understand the role 
of allochthonous inputs on recipient communities \citep{Polis1997, Mccary2020}.
One of the challenges in identifying the effects of allochthonous pulses 
from long-term observational data is that pulses are likely to be conflated
with other factors that vary through space and time \citep{Yang2008}.
By using linear mixed effects models with group-specific temporal autocorrelation, 
we were able to partition the influence of three interrelated variables (time, distance, and midge deposition) 
to quantify the effect of allochthonous resources on a tundra arthropod community.
We found a positive overall response of predator and detritivore abundance to midge deposition, which
is consistent with previous studies at M\'{y}vatn \citep{Hoekman2011, Dreyer2012, Sanchez2018}
and in other systems \citep{Ostfeld2000, Murphy2012}.
Given the ubiquity of spatiotemporal associations between different drivers of ecological communities, 
our approach has potential utility across many subdisciplines of community ecology.

