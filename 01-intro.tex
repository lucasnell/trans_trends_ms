\section*{Introduction}

A central goal of ecology is to understand how environmental variability affects
population and community dynamics.
Time series data of ecological communities collected from multiple sites across
large spatial scales are increasingly available from monitoring programs such as the
National Ecological Observatory Network \citep{Keller2008} and
long-term field experiments \citep[e.g.][]{Fraser2013, Cowles2018, Shi2015}.
Data of this sort provide great promise for answering long-standing questions about
how environmental factors drive community dynamics, but require the development of
appropriate analytical tools.

The response of ecological communities to environmental predictors is often analyzed
using ordination methods (e.g. NMDS, RDA), which map variation in abundance or
occurrence onto orthogonal axes that provide synoptic assessments of community
variation \citep{Mcgarigal2013}.
However, these methods are generally inappropriate for time-series data, as they
do not account for temporal autocorrelation in the abundance of members of the
community \citep{Ives2006}.
While some time-series methods have been developed for analyzing ecological communities,
these have generally focused on inferring interactions between species and therefore
require relatively long time series \citep{Ives1999, Hampton2013}.
In contrast, community time series are often relatively short but contain
replication through space or across experimental units, with the goal of
inferring the responses of communities to external drivers or experimental manipulations.

Here, we show how autoregressive mixed models (linear mixed models with temporal
autocorrelation structures) can be used to infer taxon- and community-level responses
to predictor variables from replicated time series data.
The approach is based on the method of \cite{Jackson2012}, who modeled variation in
taxon-specific responses to predictors as random effects.
We extend this approach for time series data by formulating models with separate
temporal autocorrelation structures for different taxonomic groups, accounting for
the fact that different taxa are likely to be characterized by different dynamics.
We then show how principle components analysis of predicted values from the model
can be used to visualize community variation along axes most associated with
environmental predictors, in a manner analogous to traditional ordination
\citep{Jackson2012}.
This approach has the advantage of making the connection between taxon- and
community-level variation explicit, while accounting for the autocorrelation
structure of time series data.

We illustrate the utility of this method by analyzing the responses of predatory
arthropods to spatiotemporal variation in allothchonous resources at Lake Mývatn
in northern Iceland \citep{Einarsson2004}.
Mývatn has large emergences of midges (Chironomidae) that subsidize the terrestrial
plant \citep{Gratton2008} and arthropod \citep{Dreyer2012, Sanchez2018} communities.
The midges have large interannual fluctuations \citep{Gardarsson2004} and decline
in deposition with distance from the lakeshore \citep{Dreyer2015}.
We apply autoregressive mixed models to assess the community response to the
highly variable midge subsidy, while also accounting for linear trends across
years and distance from the lake. This study shows how autoregressive mixed
models provide a unified approach for characterizing ecological responses to
environmental variation through time and space.


