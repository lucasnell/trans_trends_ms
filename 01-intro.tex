\section*{Introduction}

A central goal of ecology is to understand how environmental variation affects
population and community dynamics.
To this end, 
time-series data of ecological communities are increasingly available from monitoring programs such as the
National Ecological Observatory Network \citep{Keller2008} and
long-term field experiments \citep[e.g.,][]{Fraser2013, Cowles2018}.
Data of this sort provide great promise for answering long-standing questions of
how environmental factors drive community dynamics.
The responses of ecological communities to environmental predictors are often analyzed
using ordination methods (e.g., NMDS, RDA), which map variation in abundance or
occurrence onto orthogonal axes that provide synoptic assessments of community
variation \citep{Mcgarigal2013}.
However, when applied to community time series these approaches do not account for temporal
autocorrelation in the abundances of the populations that compose the community.
This is an important limitation, as failing to account for temporal autocorrelation
can lead to erroneous statistical inferences \citep{Ives2006}.
Furthermore, populations may vary in their degree of temporal autocorrelation,
due to differences in life histories or strengths of population regulation \citep{Ziebarth2010},
potentially exacerbating the challenges for statistical inference.
One approach to this problem is to use dimension reduction 
to produce composite community metrics for more traditional univariate time-series analyses \citep{Simpson2009}.
Unfortunately, this approach sacrifices valuable inferences about taxon-specific responses 
to environmental predictors.
Alternative approaches are designed to explicitly quantify dynamic interspecific interactions
but require relatively complex models and long time series \citep{Ives2003, Hampton2013}.
In contrast, community time series are often relatively short but contain
replication through space or across experimental units with the goal of
inferring the responses of communities to external drivers or experimental manipulations.

Here, we show how autoregressive mixed models (linear mixed models with temporal
autocorrelation structures) that account for observation error
can be used to infer taxon- and community-level responses
to predictor variables from replicated time series data.
The approach is based on the method of \cite{Jackson2012}, who model variation in
taxon-specific responses to predictors as random effects.
These random effects can in turn be used for statistical inferences on community composition,
which fundamentally derives from variation among taxa,
making explicit the link between the two organizational scales.
We extend this approach for time series by formulating models with separate
temporal autocorrelation structures for different taxonomic groups, accounting for
the fact that different taxa are likely to be characterized by different dynamics.
The degree of temporal autocorrelation itself provides information 
about the community dynamics \citep{Ives2003},
and its inclusion allows for valid statistical inferences on the effects of predictors 
when observations are temporally replicated \citep{Ives2006}.

We illustrate the utility of this method by analyzing the responses of 
terrestrial arthropods to spatiotemporal variation in allochthonous resources at Lake Mývatn, Iceland \citep{Einarsson2004}.
Mývatn has large emergences of midges (Diptera: Chironomidae) that emerge from the lake as adults 
and subsequently subsidize the terrestrial
plant \citep{Gratton2008, Krowiak2017} and arthropod \citep{Dreyer2012, Sanchez2018} communities.
The midges have large interannual fluctuations and decline
in deposition with distance from the lakeshore \citep{Dreyer2015}.
We apply autoregressive mixed models to assess the community response to the
highly variable midge subsidy, 
while simultaneously accounting for linear trends across time and distance from the lake.
Including those two covariates was important for isolating the effects of midges from those
of other variables that vary with distance (e.g., plant composition) or 
time (e.g., climate) \citep{Bowden2018}.
This study shows how autoregressive mixed models provide a statistically sound, 
easy-to-interpret method for characterizing community responses to
environmental variation in time and space.


