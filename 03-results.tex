\section*{Results}

The AR coefficients were generally similar across taxa and of modest magnitude,
indicating moderate levels of temporal autocorrelation across years
(Figure \ref{fig:coefs}B).
Ground spiders were an exception,
with an AR coefficient near zero indicating limited autocorrelation.
The relatively low autocorrelation is apparent from the time series,
where ground spiders appear more tightly constrained to their mean abundance
than was the case for other taxa (Figure \ref{fig:obs-data}A),
which results in less autocorrelated (i.e. ``faster'') fluctuations through time
\citep{Ziebarth2010}. 
While caution is warranted when interpreting dynamical features of short time series,
this result implies that ground spiders were more tightly constrained to their mean abundance
as determined by the environmental predictors, while the other taxa were only modestly constrained.
However, none of the AR coefficients were close to 1, 
implying that the populations were statistically stationary through time 
once accounting for the predictors, including linear time trends.



The overall community response to midge deposition was positive,
as indicated by both the coefficient estimates (Figure \ref{fig:coefs}A,C)
and the improvement of the full model
relative to the reduced model (Table \ref{tab:model-summary}: \emph{LOOIC}).
The response to midges was somewhat variable across taxa,
as judged by the taxon-specific coefficients (Figure \ref{fig:coefs}A)
and the posterior distribution
for the random effect standard deviation (Figure \ref{fig:coefs}D).
Specifically, ground spiders had the most positive response,
while sheet weavers had the weakest (largely neutral) response.  
However, this variation among taxa made a relatively small contribution
to the model fit (Table \ref{tab:model-summary}: \emph{LOOIC}).
In contrast to the response to midges, there was large variation among taxa in
linear trends through time and distance from the lake (Figure \ref{fig:coefs}A,D);
this taxon-specific variation dominated the contribution to the model fit.
Responses to distance were particularly variable, with wolf spiders 
increasing and rove beetles decreasing with distance from the lake.
Overall, the magnitudes of the taxon-specific responses to time and distance
were larger than the responses to midges.



The taxon-specific responses provided the basis for inferring variation 
in both community composition per se and overall community abundance.
Changes in composition arose when taxa differed in their responses
to predictors, while changes in overall abundance arose from consistent
responses among taxa.
For example, taxa varied in their responses to distance from the lake 
(Figure \ref{fig:coefs}; Table \ref{tab:model-summary}: \emph{LOOIC}),
which in turn implied that community composition varied with distance.
In the PCA, this resulted in variation in the taxon-response vectors
along PC2 (Figure \ref{fig:pca}A), 
which was most strongly associated with distance 
(Figure \ref{fig:pca}C; Table \ref{tab:model-summary}: \emph{Variance partitioning}).
Community composition also varied through time, 
which primarily manifested in variation along PC3 (Figure \ref{fig:pca-23}).
In contrast, the abundance of all taxa except for sheet weavers increased with midges,
resulting in an positive overall response of community abundance along PC1 (Figure \ref{fig:pca}A,D).
Distance accounted for most of the observed community variation,
followed by time, and midges (Table \ref{tab:model-summary}). 
This implies that (1) changes in composition per se
with distance and time caused most of the community variation, and 
(2) variation in overall abundance due to midges was more modest,
but still statistically meaningful.
Together, linear responses to midges, distance, and time accounted for 
68\% of the observed community variation.
