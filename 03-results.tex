\section*{Results}

The AR coefficients were generally similar across taxa and all well below 1
(Figure \ref{fig:coefs}a), 
indicating that the populations were statistically stationary
through time after accounting for the predictors (including the linear time trends).
Ground spiders had the lowest AR coefficient; 
this was visually apparent in the data,
as ground spiders appeared more tightly constrained to their mean abundance
than was the case for other taxa (Figure \ref{fig:obs-data}a).
This implies that ground spiders may have experienced the strongest population regulation
as determined by the environmental predictors,
although caution is warranted when inferring dynamical features of short time series.



Taxa varied substantially in their trends through time: 
ground spiders and beetles increased, wolf spiders and sheet weavers decreased,
while rove beetles and harvestmen remained largely unchanged (Figure \ref{fig:coefs}b).  
Variation among taxa was even greater in response to distance than for time (Figure \ref{fig:coefs}c),
with ground and rove beetles most abundant near the lake, 
while ground and wolf spiders were most abundant further away.
In contrast to time and distance,
the average response across taxa to midge deposition was consistently positive (although fairly weak).
Indeed, the responses for all individual taxa were positive, 
with only sheet weavers approaching zero (Figure \ref{fig:coefs}d).
The positive responses to midges were associated with variation in deposition 
both among years and along distance from the lake (Figure \ref{fig:obs-data}),
underscoring the importance of accounting for time and distance when inferring responses to midge deposition.



The taxon-specific responses provided the basis for inferring variation 
in both community composition per se and community-wide abundance.
Specifically, changes in composition arose when taxa differed in their responses
to predictors, while changes in community-wide abundance arose from consistent
responses among taxa.
For example, taxa differed in their responses to distance from the lake 
(``among taxa SD'' in Figure \ref{fig:coefs}c),
which in turn implied that community composition varied with distance.
In the PCA, this resulted in substantial variation among the taxon-response vectors
along PC1, which was most strongly associated with distance 
(Figure \ref{fig:pca}a--d; Table \ref{tab:model-summary}).
Taxa also differed in responses to time (``among taxa SD'' in Figure \ref{fig:coefs}b),
with variation in community composition manifesting along PC3 (Figure \ref{fig:pca}c--f).
In contrast, the average response to midges was positive (``among taxa mean'' in Figure \ref{fig:coefs}d),  
resulting in an increase in community-wide abundance along PC2 (Figure \ref{fig:pca}a,b,e,f).
Distance accounted for most of the observed community variation,
followed by midges, and finally time (Table \ref{tab:model-summary}). 
Together, linear responses of the various taxa to midges, distance, and time accounted for 
74\% of the observed community variation.
